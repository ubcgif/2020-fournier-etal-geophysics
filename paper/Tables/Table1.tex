% \documentclass[paper]{geophysics}
% \usepackage{graphicx}
% \usepackage{amsmath}
% \usepackage{amsfonts}
% \usepackage{multirow}
% \usepackage{mathtools}
% \usepackage[table]{xcolor}

% \newcommand{\crule}[3][black]{\textcolor{#1}{\rule{#2}{#3}}} % Define the \crule command
% \renewcommand{\arraystretch}{0.5}

% \definecolor{BXH}{RGB}{208, 140, 212}
% \definecolor{BXHC}{RGB}{218, 150, 222}
% \definecolor{UDU}{RGB}{230, 230, 230}
% \definecolor{IGB}{RGB}{200, 200, 200}
% \definecolor{IGP}{RGB}{128, 0, 128}
% \definecolor{IGBM}{RGB}{170, 170, 170}
% \definecolor{UPX}{RGB}{140, 140, 140}
% \definecolor{UPXO}{RGB}{140, 140, 140}
% \definecolor{UKO}{RGB}{176, 245, 184}
% \definecolor{VTU}{RGB}{0, 160, 18}
% \definecolor{MVO}{RGB}{0, 120, 0}
% \definecolor{VBA}{RGB}{0, 90, 0}
% \definecolor{VMO}{RGB}{0, 120, 0}
% \definecolor{VFO}{RGB}{178, 255, 104}
% \definecolor{SAK}{RGB}{255, 174, 0}
% \definecolor{SAN}{RGB}{255, 225, 140}
% \definecolor{MSC}{RGB}{150, 140, 140}
% \definecolor{MPHB}{RGB}{110, 190, 255}
% \definecolor{MPH}{RGB}{139, 241, 255}
% \definecolor{SOO}{RGB}{139, 241, 255}
% \definecolor{VOO}{RGB}{0, 200, 0}
% \definecolor{OVB}{RGB}{255, 255, 0}

% \begin{document}

% \begin{table}[h]
% \centering
% \begin{tabular}{|c|c|l|c|}\hline
%  & Code & Description & Susceptibility \\ \hline
% \crule[OVB]{0.25cm}{0.25cm} & OVB & Overburden & Medium \\ \hline
% \crule[IGB]{0.25cm}{0.25cm} & IGB &
% \begin{tabular}{l}
% IGB: Gabbro \\
% IPG: Pegmatite \\
% IGBO: Olivine gabbro\\
% IDI: Diorite \\
% IDB: Diabase \\
% IMO: Intrusive (mafic)\\
% IGBM: Magnetite gabbro
% \end{tabular} & \begin{tabular}{c} \\ Medium \\ \\ \\ \\ \\ \hline High \end{tabular} \\ \hline
% \crule[UPX]{0.25cm}{0.25cm}& UPX &
% \begin{tabular}{l}
% UPXO: Olivine peroxinite \\
% UOO: Ultramafic (Undiff.)\\
% MPE: Metaperidotite \\
% UWB: Websterite \\
% \end{tabular} & Medium \\ \hline
% \crule[UDU]{0.25cm}{0.25cm}& UDU &
% \begin{tabular}{l}
% UDU: Dunite \\
% UPE: Peridotite \\
% \end{tabular} & High \\ \hline
% \crule[UKO]{0.25cm}{0.25cm} & UKO & Komatiite & High \\ \hline
% \crule[BXH]{0.25cm}{0.25cm} & BXH & \begin{tabular}{l}BXO (undiff.) \\ BXHC: Hydrothermal (crackle)\end{tabular} & Medium \\ \hline
% \crule[SOO]{0.25cm}{0.25cm}& SOO &
% \begin{tabular}{l}
% MAB: Albitite \\
% MAM: Amphibolite\\
% MQZ: Quartzite \\
% MSCSD: Schist \\
% \end{tabular} & Low \\ \hline
% \crule[MPHB]{0.25cm}{0.25cm}& MPH &
% \begin{tabular}{l}
% MPH: Phyllite \\
% MPHB: Black Phyllite \\
% MSCBK: Black Schist \\
% MHF: Hornfels \\
% \end{tabular} & Low \\ \hline
% \crule[VMO]{0.25cm}{0.25cm}& MVS &
% \begin{tabular}{l}
% VMO: Volcanic Mafic \\
% VBA: Basalt\\
% VTUM: Volcanic tuff \\
% \end{tabular} & Low \\ \hline
% \crule[VOO]{0.25cm}{0.25cm}& VIO &
% \begin{tabular}{l}
% VIO: Volcanic Intermediate \\
% VTUI: Volcanic tuff \\
% VAN: Andesite \\
% VOO: Volcanic (undiff.) \\
% \end{tabular} & Medium \\ \hline
% \end{tabular}
% \end{table}


% \end{document}


\documentclass[paper]{geophysics}
\usepackage{graphicx}
\usepackage{amsmath}
\usepackage{amsfonts}
\usepackage{multirow}
\usepackage{mathtools}
\renewcommand{\arraystretch}{0.5}

\begin{document}

\begin{table}[h]
\centering
\begin{tabular}{|c|c|l|c|}
\hline
 & Code & Description & Susceptibility \\ \hline

\includegraphics[width=0.25cm,height=0.25cm]{Tables/patches/OVB.png} & OVB & Overburden & Medium \\ \hline

\includegraphics[width=0.25cm,height=0.25cm]{Tables/patches/IGB.png} & IGB &
\begin{tabular}{l}
IGB: Gabbro \\
IPG: Pegmatite \\
IGBO: Olivine gabbro \\
IDI: Diorite \\
IDB: Diabase \\
IMO: Intrusive (mafic) \\
IGBM: Magnetite gabbro
\end{tabular} &
\begin{tabular}{c} \\ Medium \\ \\ \\ \\ \\ \hline High \end{tabular} \\ \hline

\includegraphics[width=0.25cm,height=0.25cm]{Tables/patches/UPX.png} & UPX &
\begin{tabular}{l}
UPXO: Olivine peroxinite \\
UOO: Ultramafic (Undiff.) \\
MPE: Metaperidotite \\
UWB: Websterite
\end{tabular} & Medium \\ \hline

\includegraphics[width=0.25cm,height=0.25cm]{Tables/patches/UDU.png} & UDU &
\begin{tabular}{l}
UDU: Dunite \\
UPE: Peridotite
\end{tabular} & High \\ \hline

\includegraphics[width=0.25cm,height=0.25cm]{Tables/patches/UKO.png} & UKO & Komatiite & High \\ \hline

\includegraphics[width=0.25cm,height=0.25cm]{Tables/patches/BXH.png} & BXH &
\begin{tabular}{l}
BXO (undiff.) \\
BXHC: Hydrothermal (crackle)
\end{tabular} & Medium \\ \hline

\includegraphics[width=0.25cm,height=0.25cm]{Tables/patches/SOO.png} & SOO &
\begin{tabular}{l}
MAB: Albitite \\
MAM: Amphibolite \\
MQZ: Quartzite \\
MSCSD: Schist
\end{tabular} & Low \\ \hline

\includegraphics[width=0.25cm,height=0.25cm]{Tables/patches/MPHB.png} & MPH &
\begin{tabular}{l}
MPH: Phyllite \\
MPHB: Black Phyllite \\
MSCBK: Black Schist \\
MHF: Hornfels
\end{tabular} & Low \\ \hline

\includegraphics[width=0.25cm,height=0.25cm]{Tables/patches/VMO.png} & MVS &
\begin{tabular}{l}
VMO: Volcanic Mafic \\
VBA: Basalt \\
VTUM: Volcanic tuff
\end{tabular} & Low \\ \hline

\includegraphics[width=0.25cm,height=0.25cm]{Tables/patches/VOO.png} & VIO &
\begin{tabular}{l}
VIO: Volcanic Intermediate \\
VTUI: Volcanic tuff \\
VAN: Andesite \\
VOO: Volcanic (undiff.)
\end{tabular} & Medium \\ \hline

\end{tabular}
\end{table}

\end{document}
